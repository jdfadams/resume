\documentclass[11pt]{article}
\usepackage{hyperref}
\usepackage[margin=1in]{geometry}
\usepackage{paralist}
\usepackage[sc]{titlesec}

\newcommand\email{jdfadams@gmail.com}
\newcommand\phone{951-777-9170}

\title{Joseph Adams}

\begin{document}

\pagenumbering{gobble}

\begin{center}
\textsc{\Large Joseph Adams} \\[1.5\baselineskip]
\begin{tabular}{c c}
\href{mailto:\email}{\email} & \href{tel:1-\phone}{\phone} \\
\url{http://linkedin.com/in/jdfadams} & \url{http://github.com/jdfadams}
\end{tabular}
\end{center}

\section*{Employment}
\noindent\textbf{Software Engineer (remote)} \hfill December 2016 to present\\
Mathematical Sciences Publishers, Berkeley\\
\begin{inparaitem}
\item Designed, implemented, and maintained applications and APIs (Python/Django, Docker, Node.js, PHP, MySQL)
\item Modernized legacy software (PHP, Perl, Bash, Ansible)
\item Automated testing workflows (Cypress, Python/Selenium/BeautifulSoup)
\item Orchestrated connections between internal and external software services
\item Addressed customer needs
\end{inparaitem}

\noindent\textbf{Postdoctoral Fellow in Mathematics} \hfill August 2016 to December 2016\\
University of Toronto, St. George and Mississauga\\
\begin{inparaitem}
\item Performed research (quasiconformal maps, Teichm\"{u}ller spaces, dynamical systems)
\item Taught 3rd-year undergraduate complex analysis course
\item Co-organized mathematics seminar for post-docs
\end{inparaitem}

\noindent\textbf{GAANN Fellow/Research Assistant/Teaching Assistant} \hfill August 2010 to May 2016\\
Stony Brook University, New York
\begin{inparaitem}
\item Wrote dissertation establishing \textit{a priori} bounds and rigidity for infinitely primitively renormalizable polynomials with bounded combinatorics
\item Taught, led discussions, and graded undergraduate and graduate courses (calculus, linear algebra, real and complex analysis, abstract algebra)
\end{inparaitem}

\section*{Education}
\noindent\textbf{PhD in Mathematics} \hfill May 2016\\
Stony Brook University, New York\\
\noindent\textbf{BS in Mathematics} \hfill August 2010\\
University of California, Riverside\\
\noindent\textbf{Certificate in C++ Programming} \hfill August 2005\\
Riverside Community College, California

\section*{Skills}
\noindent\textbf{C++}
\begin{inparaenum}
\item implemented a genetic algorithm for the traveling salesman problem;
\item illustrated the universal approximation theorem for neural networks;
\item wrote programs for compressing files using the Huffman algorithm, evaluating mathematical expressions, and analyzing graphically the combinatorics of polynomial dynamical systems
\end{inparaenum}
\\
\noindent\textbf{Python} web exploration using Requests and Beautiful Soup; data visualization using Graphviz\\
\noindent\textbf{Machine learning} neural networks, genetic algorithms, naive Bayes, regression and SVMs\\
\noindent\textbf{\LaTeX} writing technical documents with graphics using TikZ and IPE\\
\noindent\textbf{Mathematics} linear algebra, persistent homology, probability, quasiconformal maps, hyperbolic geometry, holomorphic dynamics, and Teichmüller theory\\
\noindent\textbf{other technical} I am familiar with pandas and MapReduce, and I have used Maxima, Sage, Octave/MATLAB, Mathematica, HTML, CSS, and PHP.\\
\noindent\textbf{personal}
\begin{inparaitem}
\item I co-organized the Dynamical Systems Seminar at Univesity of Toronto (Fall 2016) and the Graduate
Student Seminar at Stony Brook University (Fall 2013 to Spring 2014).
\item I taught t’ai chi ch’uan as a
course for Riverside Community College and in seminars for the Multiple Sclerosis Society (Spring
2007 to Fall 2008).
\item I climbed Denali (Summer 2009).
\end{inparaitem}

\end{document}
